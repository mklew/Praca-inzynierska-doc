\pdfoutput=1
\pdfcompresslevel=9
\pdfinfo
{
    /Author (Autor)
    /Title (Tytul)
    /Subject (Tematyka)
    /Keywords (Slowa kluczowe)
}
%\documentclass[a4paper,polish,onecolumn,oneside,floatssmall,11pt,titleauthor,wide,openright]{mwrep}
%\usepackage[scale={0.7,0.8},paper=a4paper,twoside]{geometry}
\documentclass[a4paper,onecolumn,oneside,11pt,wide,floatssmall]{mwrep}
% \usepackage{polish}
\usepackage{amsmath}
\usepackage{amsfonts}
\usepackage{amssymb}
\usepackage{amsthm}
\usepackage{bookman}

\usepackage{geometry}
\usepackage[utf8x]{inputenc}
\usepackage[T1]{fontenc}
% \usepackage{t1enc}
% \usepackage[pdftex, bookmarks]{hyperref}
\usepackage[pdftex, bookmarks=false]{hyperref}
\def\url#1{{ \tt #1}}

\usepackage{listings}
\usepackage{float}

% marginesy
\textwidth\paperwidth
\advance\textwidth -55mm
\oddsidemargin-0.9in
\advance\oddsidemargin 33mm
\evensidemargin-0.9in
\advance\evensidemargin 33mm
\topmargin -1in
\advance\topmargin 25mm
\setlength\textheight{48\baselineskip}
\addtolength\textheight{\topskip}
\marginparwidth15mm

\clubpenalty=10000 % to kara za sierotki
\widowpenalty=10000 % nie pozostawia wdów
\brokenpenalty=10000 % nie dzieli wyrazów pomiędzy stronami
\sloppy

\tolerance4500
\pretolerance250
\hfuzz=1.5pt
\hbadness1450

% ŻYWA PAGINA
\renewcommand{\chaptermark}[1]{\markboth{\scshape\small\bfseries \
#1}{\small\bfseries \ #1}}
\renewcommand{\sectionmark}[1]{\markboth{\scshape\small\bfseries\thesection.\
#1}{\small\bfseries\thesection.\ #1}}
\newcommand{\headrulewidth}{0.5pt}
\newcommand{\footrulewidth}{0.pt}
\pagestyle{uheadings}

\usepackage[pdftex]{color,graphicx}
\usepackage[polish]{babel}

% \textheight232mm
% \setlength{\textwidth}{\textwidth}
% \setlength{\oddsidemargin}{\evensidemargin}
% \setlength{\evensidemargin}{0.3cm}
\usepackage[sort, compress]{cite}

%\usepackage{multibib}
%\newcites{bk,st,doc,web}{Książki i~artykuły,Standardy i~zalecenia,Dokumentacja produktów,Publikacje i~serwisy internetowe}

\theoremstyle{definition}
\newtheorem{defn}{Definicja}[section]
\newtheorem{conj}{Teza}[section]
\newtheorem{conjmain}{Teza}
\newtheorem{exmp}{Przykład}[section]

\theoremstyle{plain}% default
\newtheorem{thm}{Twierdzenie}[section]
\newtheorem{lem}[thm]{Lemat}
\newtheorem{prop}[thm]{Hipoteza}
\newtheorem*{cor}{Wniosek}

\theoremstyle{remark}
\newtheorem*{rem}{Uwaga}
\newtheorem*{note}{Uwaga}
\newtheorem{case}{Przypadek}

\definecolor{ListingBackground}{rgb}{0.95,0.95,0.95}

\begin{document}

% kody źródłowe wplatane w tekst
\lstdefinestyle{incode}
{
basicstyle={\footnotesize},
keywordstyle={\bf\footnotesize\color{blue}},
commentstyle={\em\footnotesize\color{magenta}},
numbers=left,
stepnumber=5,
firstnumber=1,
numberfirstline=true,
numberblanklines=true,
numberstyle={\sf\tiny},
numbersep=10pt,
tabsize=2,
xleftmargin=17pt,
framexleftmargin=3pt,
framexbottommargin=2pt,
framextopmargin=2pt,
framexrightmargin=0pt,
showstringspaces=true,
backgroundcolor={\color{ListingBackground}},
extendedchars=true,
% title=\lstname,
captionpos=b,
% abovecaptionskip=1pt,
% belowcaptionskip=1pt,
frame=tb,
framerule=0pt,
}

% kody źródłowe z podpisem
\lstdefinestyle{outcode}
{
basicstyle={\footnotesize},
keywordstyle={\bf\footnotesize\color{blue}},
commentstyle={\em\footnotesize\color{magenta}},
numbers=left,
stepnumber=5,
firstnumber=1,
numberfirstline=true,
numberblanklines=true,
numberstyle={\sf\tiny},
numbersep=10pt,
tabsize=2,
xleftmargin=17pt,
framexleftmargin=3pt,
framexbottommargin=2pt,
framextopmargin=2pt,
framexrightmargin=0pt,
showstringspaces=true,
backgroundcolor={\color{ListingBackground}},
extendedchars=true,
% title=\lstname,
captionpos=b,
% abovecaptionskip=1pt,
% belowcaptionskip=1pt,
frame=tb,
framerule=0.1pt,
}

\renewcommand*\lstlistingname{Wydruk}
\renewcommand*\lstlistlistingname{Spis wydruków}

\pagenumbering{roman}
\renewcommand{\baselinestretch}{1.0}
\raggedbottom

\begin{titlepage}
    % Strona tytułowa
    \vbox to\textheight{\hyphenpenalty=10000
    \begin{center}
	\begin{tabular}{p{107mm} p{9cm}}
	    \begin{minipage}{9cm}
	      \begin{center}
	      Politechnika Warszawska \\
	      Wydział Elektroniki i~Technik Informacyjnych \\
	      Instytut Informatyki
	      \end{center}
	    \end{minipage}
	    &
	    \begin{minipage}{8cm}
	    \begin{flushleft}
	     \footnotesize
	      Rok akademicki 2012/2013
	    \vspace*{2.75\baselineskip}
	    \end{flushleft}
	    \end{minipage} \\
	\end{tabular}
	\vspace*{3.75\baselineskip}
	\par\vspace{\smallskipamount}
	\vspace*{2\baselineskip}{\LARGE PRACA DYPLOMOWA INŻYNIERSKA\par}
	\vspace{3\baselineskip}{\LARGE\strut Marek Lewandowski\par}
	\vspace*{2\baselineskip}{\huge\bfseries System zarządzania obłożeniem hotelu - aplikacja w architekturze trójwarstwowej\par}

	\vspace*{7\baselineskip}
	\hfill\mbox{}\par\vspace*{\baselineskip}\noindent
	\begin{tabular}[b]{@{}p{3cm}@{\ }l@{}}
	    {\large\hfill } & {\large }
	\end{tabular}
	\hfill
	\begin{tabular}[b]{@{}l@{}}
	Opiekun pracy: \\[\smallskipamount]
	{\large dr inż. Jarosław Dawidczyk}
	\end{tabular}\par
	\vspace*{4\baselineskip}
    \begin{tabular}{p{\textwidth}}
    \begin{flushleft}
	\begin{minipage}{7cm}
	Ocena \dotfill
	\par\vspace{1.6\baselineskip}
	\dotfill
	\par\noindent
	\centerline{\footnotesize Podpis Przewodniczącego} \par
	\centerline{\footnotesize Komisji Egzaminu Dyplomowego}\par
	\end{minipage}
    \end{flushleft}
    \end{tabular}
    \end{center}}

    % Życiorys
    \newpage\thispagestyle{empty}
    \begin{tabular}{p{5cm} p{12cm}}
    \begin{minipage}{5cm}
    \center
    \includegraphics[height=6.5cm,width=4.5cm]{../img/foto.jpg}
    \end{minipage}
    &
    \begin{minipage}{12cm}
    \begin{flushleft}
    \par\noindent\vspace{1\baselineskip}
    \begin{tabular}[h]{l l}
    {\normalsize\it Specjalność:} & Inżynieria Systemów Informatycznych
    \end{tabular}
    \par\noindent\vspace{1\baselineskip}
    \begin{tabular}[h]{l l}
    {\normalsize\it Data urodzenia:} & {\normalsize 1 stycznia 1990~r.}
    \end{tabular}
    \par\noindent\vspace{1\baselineskip}
    \begin{tabular}[h]{l l}
    {\normalsize\it Data rozpoczęcia studiów:} & {\normalsize 1 października 2009 r.}
    \end{tabular}
    \par\noindent\vspace{1\baselineskip}
    \end{flushleft}
    \end{minipage}
    \end{tabular}
    \vspace*{1\baselineskip}
    \begin{center}
	{\large\bfseries Życiorys}\par\bigskip
    \end{center}

    \indent
    Urodziłem się 7 stycznia 1990 roku w Warszawie. W latach 2006-2009 uczęszczałem do I Liceum Ogólnokształcącego 
    im. Wacława Nałkowskiego w Wołominie. W roku 2009 zdałem egzamin maturalny i rozpocząłem studia na wydziale Elektroniki i Technik Informacyjnych Politechniki Warszawskiej.

    \par
    \vspace{2\baselineskip}
    \hfill\parbox{15em}{{\small\dotfill}\\[-.3ex]
    \centerline{\footnotesize podpis studenta}}\par
    \vspace{3\baselineskip}
    \begin{center}
 	{\large\bfseries Egzamin dyplomowy} \par\bigskip\bigskip
    \end{center}
    \par\noindent\vspace{1.5\baselineskip}
    Złożył egzamin dyplomowy w dn. \dotfill
    \par\noindent\vspace{1.5\baselineskip}
    Z wynikiem \dotfill
    \par\noindent\vspace{1.5\baselineskip}
    Ogólny wynik studiów \dotfill
    \par\noindent\vspace{1.5\baselineskip}
    Dodatkowe wnioski i uwagi Komisji \dotfill
    \par\noindent\vspace{1.5\baselineskip}
    \dotfill

    % Streszczenie
    \newpage\thispagestyle{empty}
    \vspace*{2\baselineskip}
    \begin{center}
	{\large\bfseries Streszczenie}\par\bigskip
    \end{center}

    {\itshape
    Celem pracy było stworzenie systemu zarządzania obłożeniem hotelu. Pierwszy rozdział zawiera cel pracy inżynierskiej. W drugim rozdziale omówiona została dziedzina problemu. Zawarte w nim informacje dotyczą
    organizacji hotelu, rezerwacji hotelowej i innych.
    W rozdziale trzecim przedstawione zostały wymagania systemu. Kolejny rozdział zawiera informacje na temat wykorzystanych technologii wraz z dostępnymi alternatywami oraz wytłumaczenie dlaczego konkretne technologie zostały wybrane. W rozdziale tym omówiona została także architektura trójwarstwowa. Rozdział piąty opisuje model danych. Rozdział szósty zawiera informacje dotyczące implementacji systemu. Ostatni rozdział podsumowuje pracę.
    }
    \vspace*{1\baselineskip}

    \noindent{\bf Słowa kluczowe}: {\itshape architektura trójwarstwowa, rezerwacja, hotel}
    \par
    \vspace{4\baselineskip}
    \begin{center}
	{\large\bfseries Abstract}\par\bigskip
    \end{center}
    \noindent{\bf Title}: {\itshape Hotel management system - application based on three-tier architecture}\par
    \vspace*{1\baselineskip}
    {\itshape
    This thesis describes \ldots}
    \vspace*{1\baselineskip}

    \noindent{\bf Key words}: {\itshape key words.}

\end{titlepage}

% ex: set tabstop=4 shiftwidth=4 softtabstop=4 noexpandtab fileformat=unix filetype=tex spelllang=pl,en spell:


\tableofcontents
% \addcontentsline{toc}{chapter}{{Przedmowa1}{vii}}{vii}

% \chapter*{Spis tablic, rysunków i~wydruków}
% \listoftables
% \listoffigures
% \lstlistoflistings

%\setlength{\baselineskip}{7mm}
\newpage
\pagenumbering{arabic}
\setcounter{page}{1}

% \input {chap1}
%%%% ROZDZIAŁ PIERWSZY %%%%%%%

\chapter{Analiza dziedziny} 
TODO opis tego co znajduje się w tym rozdziale, max 5 zdań


\section{Źródła informacji}

Analiza dziedziny powstała na wskutek agregacji i uporządkowania informacji 
z kilku źródeł. Posłużyłem się literaturą branżową, ustawą i informacjami z 
internetu jak również uzyskałem informację przeprowadzając rozmowy z 
pracownikami hotelu na różnym szczeblu. Miałem okazję rozmawiać z 
recepcjonistką oraz kierownikiem recepcji w dwóch różnych hotelach w 
Warszawie. Poniżej czytelnik znajdzie przefiltrowane, spójne wiadomości na
 temat branży hotelarskiej.


\section{Historia hotelarstwa} 

Ludzie zmieniali swoje miejsce od tysiącleci. Dlaczego? Powody były różne i 
przez tysiąclecia znacząco się nie zmieniły. Kiedyś wędrówki handlowe, a 
dziś wyjazdy biznesowe. Religijne wyprawy w miejsca kultu, podróże 
turystyczne do atrakcyjnych, bądź historycznych miejsc.     Historia 
hotelarstwa ma bogatą przeszłość i sięga aż II tysiąclecia p.n.e gdzie na 
terenach Bliskiego Wschodu znaleziono pierwsze ślady budownictwa 
nastawionego na gościnę podróżnych. Najstarsze domy zajezdne odnotowano 
wzdłuż szlaków handlowych w miejscach obfitujących w wodę pitną. W 
starożytnej Grecji i Rzymie budowano zajady w miejscach kultu religijnego, 
bądź miejsc odbywania się igrzysk. W czasach Średniowiecza gościny udzielano 
w klasztorach, początkowo nieodpłatnie, z czasem jednak gościna przyjęła 
formę rynkową. Edykt Karola Wielkiego nałożył na klasztory i kościoły 
obowiązek utrzymania hospicjów, gdzie podróżnym udzielano wyżywienia, 
kąpieli i opieki medycznej. Najgęstsza sieć hospicjów znajdowała się na 
terenie dzisiejszej Szwajcarii. Szwajcaria posiada najdłuższe tradycje 
hotelarskie i jest uważana za wzór hotelarstwa takiego jak znamy dzisiaj.


\section{Hotelarstwo i usługa hotelarska}

\subsection{Czym charakteryzuje się hotelarstwo?}

Hotelarstwo to forma działalności gospodarczej charakteryzująca się:
\begin{itemize} 
  \item szczególnym rodzajem gościnności - gościnność za odpłatnością,
  \item na działalność hotelu składają się różnego rodzaju usługi. Główne z 
  nich to usługi bytowe takie jak zapewnienie noclegu oraz wyżywienia,
  \item pobyt gości w hotelach jest z reguły określony w czasie i 
  krótkotrwały,
  \item zakres usług świadczonych przez hotelarzy powiększa się i integruje 
  z innymi działalnościami turystycznymi i rozrywkowymi.
\end{itemize}

\subsection{Definicja usługi hotelarskiej}

Czym jest usługa hotelarska? Na to pytanie odpowie treść ustawy o usługach 
turystycznych \cite{ust:tur}.

\begin{defn}{Usługa hotelarska}

krótkotrwałe, ogólnie dostępne wynajmowanie domów, mieszkań, pokoi, miejsc 
noclegowych, a także miejsc na ustawienie namiotów lub przyczep 
samochodowych oraz świadczenie, w obrębie obiektu, usług z tym związanych.

\end{defn}

Usługi hotelarskie zaspokajają więcej potrzeb niż może to wynikać z 
definicji, a są to: wypoczynek, pożywienie, nocleg, higiena, rekreacja, 
rozrywka, zapewnienie bezpieczeństwa, opieka nad zdrowiem, zapewnienie 
wygody, dobrej atmosfery pobytu i inne drobne usługi.

Pewne usługi są regulowane prawnie, poprzez przepisy dotyczące wymagań 
rodzajowych i kategoryzacyjnych. Jednakże abstrahując od przepisów można 
stwierdzić iż w interesie każdego hotelarza jest zapewnienie tak bogatej 
palety usług jak to możliwe, aby wyróżniać się na rynku i przyciągać gości. 
Spełnienie minimum wymaganego prawnie nie jest szczególnie atrakcyjne z 
punktu widzenia gościa. Więcej o usługach dowiemy się wkrótce.

\section{Klasyfikacja i kategoryzacja obiektów hotelarskich}
Klasyfikacja określa podział obiektów na rodzaje. Rodzaje obiektów różnią 
się miedzy sobą zakresem działalności.

Definicje obiektów określa Rozporządzenie Ministra Gospodarki i Pracy z dnia 
19 sierpnia 2004 roku, oto część z nich:

\begin{itemize}
  \item hotele - obiekty posiadające co najmniej 10 pokoi, w tym większość 
  miejsc w pokojach jedno- i dwuosobowych, świadczące szeroki zakres usług 
  związanych z pobytem klientów;
  \item motele - obiekty położone przy drogach, dysponujące parkingiem, 
  posiadające co najmniej 10 pokoi, w tym większość miejsc w pokojach jedno- 
  i dwuosobowych;
  \item pensjonaty - obiekty posiadające co najmniej 7 pokoi, świadczące dla 
  swoich klientów całodzienne wyżywienie;
  \item kempingi - obiekty strzeżone, umożliwiające nocleg w namiotach i 
  przyczepach samochodowych, domkach turystycznych lub innych obiektach 
  stałych, oraz przyrządzanie posiłków i parkowanie samochodów.
\end{itemize}

Rozporządzenie definiuje także domy wycieczkowe, schroniska młodzieżowe, 
schroniska i pola biwakowe. Wiele obiektów nie zostało zdefiniowanych, m.in. 
ośrodki wczasowe, kwatery prywatne, internaty, domy studenckie.


\subsection{Podział według kategorii} 
Obiekty hotelarskie można podzielić według różnych kategorii\cite[15-16]{
KlasKatZakHot}:

\begin{enumerate}
  \item według lokalizacji
    \begin{itemize}
      \item hotele miejskie
      \item obiekty wypoczynkowe, sportowe, kuracyjne
      \item motele - przy szlakach komunikacyjnych
    \end{itemize}
  \item według średniego pobytu gościa w obiekcie
    \begin{itemize}
      \item hotele i pensjonaty stale lub prawie stale czynne
      \item zakłady rezydenckie (minimum 10-15 dni)
      \item hotele dla przejezdnych (2-3 dni)
    \end{itemize}
  \item według zakresu świadczonych usług
    \begin{itemize}
      \item hotel pełny - pełny zakres usług
      \item hotel pensjonat - brak restauracji dla gości z zewnątrz, brak 
      napojów alkoholowych
      \item hotel garni - tylko wydawanie śniadań i prowadzenie bufetu
    \end{itemize}
  \item według okresu eksploatacji
    \begin{itemize}
      \item stałe
      \item sezonowe
    \end{itemize}
  \item według siedziby obiektu
    \begin{itemize}
      \item stałe
      \item ruchome - zakłady w okresie ferii, domki kempingowe, wagony 
      sypialne, samoloty
    \end{itemize}
  \item według dostępności
    \begin{itemize}
      \item otwarte - ogólnodostępne
      \item półotwarte - domy wczasowe, domy zdrojowe
      \item zamknięte - tylko dla określonych grup ludności - domy 
      akademickie, sanatoria
    \end{itemize}
\end{enumerate}

\subsection{Wymagania kategoryzacyjne}
Wymagania dla obiektów kategoryzowanych określa to samo rozporządzenie. 
Ustalono w nim następujące kategorie bazy noclegowej:

\begin{itemize}
  \item dla hoteli, moteli i pensjonatów - pięć kategorii oznaczonych 
  gwiazdkami,
  \item dla kempingów - cztery kategorie oznaczone gwiazdkami,
  \item dla domów wycieczkowych i schronisk młodzieżowych - trzy kategorie 
  oznaczane cyframi rzymskimi.
\end{itemize}

Kategoryzacja niesie ze sobą wymagania. Każda kategoria ma ściśle określone 
kryteria, które obiekt musi spełniać. Kryteria te określają minimalny poziom 
świadczonych usług. Wprowadzają też porządek do słownictwa i umożliwiają 
standaryzację obiektów. Poświęcę szczególną uwagę pięcio gwiazdkowej 
kategoryzacji.

\subsection{Pięć gwiazdek}
Prawdopodobnie wszyscy znają ten podział, a przynajmniej wiedzą, że im 
więcej gwiazdek tym wyższy standard. Różne turystyczne oferty mamią nas 
gwiazdkami. Warto wiedzieć na co faktycznie możemy liczyć, ale czy na pewno 
nasze oczekiwania zostaną spełnione? Kategoryzacja istnieje w wielu krajach 
europejskich, ale jedna jest drugiej nie równa. Szczegółowość wymagań dla 
kategorii różni się pomiędzy krajami. Nie wszędzie jest to także regulowane 
prawnie. W Niemczech nie ma kategoryzacji, a kategorie obiektów są stosowane 
w sposób umowny przez wydawców przewodników i biura podróży. Warto mieć to 
na uwadze przy następnej podróży.

\section{Podział usług hotelarskich}   
Omówię podział usług hotelarskich, a następnie krótko je zcharakteryzuję na 
podstawie \cite[8-10]{OrgaDzialHot}.

Podstawowymi usługami świadczonymi przez hotel są usługi noclegowe i 
gastronomiczne. Wszystkie inne usługi są im podporządkowane.

Wszystko co wykracza poza usługi podstawowe można podzielić w następujący 
sposób:
  \begin{itemize}
    \item usługi uzupełniające
    \item usługi fakultatywne
    \item usługi towarzyszące
  \end{itemize}

\subsection{Usługi uzupełniające}
Usługi uzupełniające uzupełniają podstawowe usługi i są z nimi bezpośrednio 
związane. Gość jest zmuszony z nich korzystać w momencie gdy decyduje się na 
pobyt w hotelu. Przykładami usług uzupełniających są:

\begin{itemize}
  \item budzenie
  \item szatnia
  \item depozyt
  \item informacja turystyczna
\end{itemize}

Usługi uzupełniające są zazwyczaj bezpłatne, ponieważ są wliczone w cenę 
usług podstawowych.

\subsection{Usługi fakultatywne}
Kolejny rodzaj usług to usługi fakultatywne. Te usługi są już mniej 
powiązane z usługami podstawowymi. Usługi fakultatywne uatrakcyjniają pobyt 
gościom. Przykłady usług:

\begin{itemize}
  \item basen
  \item sauna
  \item tenis
  \item rozrywka
\end{itemize}

Podczas pobytu korzystanie z nich jest opcjonalne, ale zazwyczaj trzeba być 
gościem hotelu, żeby móc z nich skorzystać. To czy te usługi są płatne czy 
nie zależy od hotelu.  Jeśli są bezpłatne, stanowią zachętę dla gości (mimo 
tego, że mogą być wliczone w cenę).

\subsection{Usługi towarzyszące}
Usługi towarzyszące są całkowicie ortogonalne w stosunku do usług 
podstawowych. Oznacza to, że nie trzeba być hotelowym gościem, aby móc z 
nich skorzystać. Tak samo jak usługi fakultatywne, usługi towarzyszące są 
opcjonalne. Usługi towarzyszące to odrębny rodzaj działalności z lokalizacją 
przy hotelu. Przykładami usług towarzyszących są:

\begin{itemize}
  \item kiosk
  \item kwiaciarnia
  \item kasyno
  \item usługi handlowe
\end{itemize}

\subsection{Usługi obce}
Te usługi, które są świadczone bezpośrednio przez obsługę hotelową nazywa 
się usługami własnymi. Są jednak pewne usługi dodatkowe, których hotel nie 
może, bądź nie jest to opłacalne, świadczyć. Sprzedawanie usług dodatkowych 
korzystając z zewnętrznych firm pozwala hotelowi zwiększyć swoją atrakcyjność
. Przykładem usług obcych mogą być wszelakiej maści usługi turystyczne, 
które organizuje zewnętrzne biuro turystyczne.

Usługi obce to także dostarczanie energii elektrycznej, wody i innych mediów.
 Tutaj hotel jest zdany na innych, ponieważ samemu nie może wszystkiego 
wytworzyć.

\subsection{Wymagania kategoryzacyjne, a usługi}
Część usług dodatkowych może być wymuszona poprzez wymagania kategoryzacyjne.
 Te usługi są \emph{obowiązkowe}. Reszta usług dodatkowych, świadczonych 
 przez hotel dla wygody gości i podniesienia prestiżu 
 hotelu, to usługi \emph{dobrowolne}.

\section{Istotne definicje}
Przedstawię definicję noclegu oraz doby hotelowej. Następnie przedstawię 
definicję klienta i agenta. Więcej o klientach dowiemy się w
 \ref{rodzaje-klientow}.

\subsection{Definicja noclegu}

\begin{defn}{Nocleg}

Usługa noclegowa dla jednej osoby w wymiarze jednej doby hotelowej tzw. 
osobodzień. Nocleg jest jednostką miary usługi noclegowej.

\end{defn} 

Nocleg wykorzystywany jest w miernikach usług hotelarskich:
\begin{itemize}
  \item nominalna liczba noclegów - ilość miejsc, które można wynająć w 
  danym okresie.
  \item wskaźnik frekwencji - stopień wykorzystania miejsc
\end{itemize}

\subsubsection{Wskaźnik RevPAR}
Podstawowym wskaźnikiem jest wskaźnik RevPAR\footnote{skrót: Revenue per 
available room}. RevPAR określa zrealizowane, 
dzienne obroty każdego wybudowanego pokoju. Liczony jest na podstawie
 ilorazu dochodów z pokojów do dostępnych pokojów. RevPAR uwzględnia tylko 
 dochód z wynajmu pokojów. Nie brane są pod uwagę inne źródła dochodu.

\begin{equation}
RevPAR = \frac{Rooms Revenue}{Rooms Available}
\end{equation}

Może być także oszacowany w następujący sposób:

\begin{equation}
RevPAR \approxeq Freq \cdot ADR
\end{equation}

\begin{description}
  \item[Freq] wskaźnik frekwencji
  \item[ADR]\footnote{ang. {\em Average Daily Rate} } średnia cena pokoi

\end{description}

TODO TODO TODO napisać jak liczone jest Freq i ADR

\subsection{Definicja doby hotelowej}
Doba hotelowa różni się od zwykłej doby.

\begin{defn}{Doba hotelowa} - trwa do godziny określonej jako godzina 
zakończenia doby 
hotelowej. Godzina zakończenia doby hotelowej jest ustalana przez hotel.
 Długość doby hotelowej dla gościa może być większa lub mniejsza 
od normalnej doby. Doba hotelowa została wprowadzona ze względu na 
organizację pracy hotelu, m.in. sprzątania pokojów.

\end{defn}

Załóżmy, że godzina zakończenia doby hotelowej to 12.00. Jeśli gość 
przyjechał o godzinie 10 to będzie mógł korzystać z pokoju do godziny 12 
następnego dnia, a więc więcej niż 24 godziny. Przyjeżdżając po godzinie 12 
będzie mógł korzystać z pokoju krócej. W obu przypadkach gość zapłacił tyle 
samo za pojedynczą dobę hotelową.


\subsection{Klient}

Klient, bądź gość również został zdefiniowany w ustawie\cite{ust:tur}:

\begin{defn}{Klient}
- osoba, która zamierza zawrzeć lub zawarła umowę o świadczenie usług 
turystycznych na swoją rzecz lub na rzecz innej osoby, a zawarcie tej umowy 
nie stanowi przedmiotu jej działalności gospodarczej, jak i osobę, na rzecz 
której umowa została zawarta, a także osobę, której przekazano prawo do 
korzystania z usług turystycznych objętych uprzednio zawartą umową.

\end{defn}

Definicja bazująca na ustawie jest dość skomplikowana, więc można spróbować 
prościej:

\begin{defn}{Klient}
- osoba, korzystająca z usług hotelarskich, w szczególności usługi noclegowej
.

\end{defn}

Klient oraz gość nie znaczą dokładnie tego samego, ponieważ jak wiemy, 
hotel świadczy usługi towarzyszące, których świadczeniobiorca nie musi być 
gościem hotelowym. Jednak nie wdając się w szczegóły, 
będziemy używać tych nazw wymiennie. Więcej o klientach dowiemy się w 
\ref{rodzaje-klientow}

\subsection{Agent}
Agent biura turystycznego pełni rolę pośrednika. Zazwyczaj biuro turystyczne 
wykupuje pewny pakiet noclegów w hotelu. Kupując hurtowo otrzymuje niższe 
ceny, niż zwykli klienci, a zatem może pozwolić sobie na ciekawszą ofertę. 
Oto definicja agenta\cite{ust:tur}:

\begin{defn}{Agent turystyczny}
przedsiębiorca, którego działalność polega na stałym pośredniczeniu w 
zawieraniu umów o świadczenie usług turystycznych na rzecz organizatorów 
turystyki posiadających zezwolenia w kraju lub na rzecz innych usługodawców 
posiadających siedzibę w kraju.
\end{defn}

\section{Organizacja w zakładzie hotelarskim}
Przyjrzymy się teraz organizacji w zakładzie hotelarskim. Organizacja w 
zakładzie hotelarskim zależy od typu zakładu hotelarskiego. 

Inaczej będzie wyglądać organizacja w małym pensjonacie, prowadzonym przez 
kilka osób (np. rodzinę), a inaczej wyglądać będzie organizacja dużego 
luksusowego hotelu nastawionego na klientów biznesowych. 

W pierwszym przypadku członkowie rodziny i kilku pracowników mają szeroki 
zakres obowiązków i oczekiwane jest wykonywanie działań, które aktualnie są 
potrzebne. Struktura organizacja będzie dość prosta i płaska.

W przypadku znacznie większego hotelu, który zatrudnia znaczącą liczbę 
pracowników i obsługuje kilkuset gości wymagana będzie bardziej rozbudowana 
struktura organizacyjna. Przyjrzymy się podobnemu przypadkowi z bliska i 
poznamy jak może wyglądać taka struktura. 

Zanim jednak do tego przejdziemy zapoznajmy się z definicją struktury.

\subsection{Definicja struktury}
Definicja na podstawie \cite{bk:def-struktury}:
\begin{defn}{Struktura} 
to instrument zarządzania organizacją. To uporządkowanie pozycji w 
organizacji według zasad podziału pracy, co wiąże się z określeniem 
specjalistycznych zadań, które muszą być wykonane, aby organizacja mogła 
funkcjonować i osiągać założone cele. To zgrupowanie ról organizacyjnych w 
większe całości, podsystemy: sekcje, wydziały, zakłady, piony - gdzie 
następuje obarczenie odpowiedzialnością za sprawną realizację zagregowanych 
zadań poszczególnych komórek organizacyjnych, a zwłaszcza kierowników tych 
podsystemów. To w końcu podział władzy i informacji w ramach organizacji, a 
więc uporządkowanie pionowego podziału pracy w ramach szczebli struktury(
poziomów jej hierarchii)
\end{defn}

\subsection{Struktura organizacyjna w średniej wielkości hotelu}
Pod lupę weźmiemy średniej wielkości hotel. Na tyle duży, że wymaga 
wprowadzenia struktury organizacyjnej, ale na tyle skomplikowanej, że można 
będzie ją przedstawić w prosty sposób.

\subsubsection{Przykład struktury organizacyjnej hotelu}
Przykładową strukturę organizacyjną przedstawia poniższy rysunek 
\ref{fig:struktura-hotelu}.

\begin{figure}[H]
  \begin{center}
  \fbox{
    \includegraphics[width=\textwidth]{../img/struktura.png}
  }
  \end{center}
  \caption{Przykładowa struktura hotelu}
  \label{fig:struktura-hotelu}
\end{figure}

\subsubsection{Charakterystyka przykładowej struktury}
Na szczycie znajduje się dykrektor, ale bardziej interesujący jest dalszy 
podział. Struktura zawiera w sobie dwie struktury:
\begin{itemize}
  \item struktura produkcyjna - zaliczają się do niej wszystkie działy i 
  komórki, które w sposób bezpośredni związane są z \emph{produkcją}
  jednostki mieszkalnej gotowej do sprzedaży, sali konferencyjnej do 
  wynajmu, posiłku do wydania;
  \item struktura administracyjna - do której należą wszystkie działy, które
   nie są w strukturze produkcyjnej. Na rysunku pominąłem dział księgowy, 
   który jak najbardziej istnieje i zalicza się do struktury administracyjnej
   .
\end{itemize}

Struktura podzielona jest na 4 działy:
\begin{itemize}
  \item Dział Hotelarski - omówiony poniżej
  \item Dział Gastronomiczny - odpowiada za przygotowanie posiłków i napoi 
  do sprzedaży. Zapewnia gościom hotelowym wyżywienie w zakresie co najmniej 
  śniadania,
  \item Dział Marketingu i Sprzedaży - realizuje działania w efekcie, 
  których klient wybiera ten, a nie inny hotel,
  \item Dział Administracyjny - finanse, remonty, usługi wewnętrzne, 
  koordynacja działania przedsiębiorstwa
\end{itemize}

Działania tych działów muszą być koordynowane, tak aby każdy z nich 
efektywnie uczestniczył w zaspokajaniu potrzeb gości.

Za każdy dział odpowiedzialny jest kierownik, który odpowiada przed 
dyrektorem. Dyrektor nadzoruje i koordynuje działania kierowników.

\subsection{Dział Hotelarski}
W naszym przykładzie dział hotelarski składa się z:
\begin{itemize}
  \item węzeła recepcyjnego - scharakteryzowanego poniżej,
  \item służba pięter - odpowiada za utrzymanie czystości w hotelu oraz za 
  przygotowanie jednostek mieszkalnych do sprzedaży oraz ciągów 
  komunikacyjnych do użytkowania.
\end{itemize}

Wymienione wyżej działy to tylko przykład, a prawdziwa struktura 
organizacyjna może być o wiele bardziej skomplikowana.

Przyjrzyjmy się bliżej węzłowi recepcyjnemu.

\subsection{Węzeł recepcyjny}
Recepcja pełni ważną rolę w strukturze organizacyjnej. Od strony gościa, 
można powiedzieć, że nawet najważniejszą. 
Recepcja obsługuje klienta od momentu przyjęcia rezerwacji, w ciągu pobytu, 
aż do zakończenia pobytu i opuszczenia przez gościa hotelu. Recepcja spełnia 
także funkcje marketingowe 
oraz sprzedażowe. Posiada także funkcje zakwaterowania i obsługi gościa 
podczas pobytu. Jako, że sam proces zakwaterowania jest bardzo krótki i trwa 
kilka, kilkanaście minut można powiedzieć, że najważniejszą i główną funkcją 
węzła recepcyjnego jest obsługa gościa w trakcie jego pobytu.

Recepcja pełni także funkcję sprzedażową, ponieważ klient, 
który przyjdzie wprost z ulicy chcąc wynająć pokój skieruje się recepcji. 
Jest to zatem  bardzo ważna komórka i jedna z pierwszych, z którą spotyka 
się gość podczas swojego pobytu. Oprócz tego pracownicy recepcji, mogą 
oferować płatne usługi, co także tłumaczy funkcję sprzedażową.

Potocznie mówi się, że pierwsze wrażenie jest najważniejsze. Recepcja jest 
dobrym tego przykładem. To od pracowników recepcji zależy, jakie będzie 
pierwsze wrażenie na kliencie. Od jakości obsługi podczas pobytu zależeć 
będzie czy pierwsze wrażenie zostanie utrzymane na dobrym poziomie oraz to 
czy klient opuści hotel szczęśliwy i będzie chciał wkrótce wrócić czy też nie
.


\subsubsection{Pracownicy recepcji}
W recepcji możemy spotkać \cite[47-51]{OrgaDzialHot}:

\begin{itemize}
  \item kierownika recepcji (front office manager) - odpowiada za sprawne 
  funkcjonowanie recepcji,
  \item recepcjonistę dysponenta - do jego zadań należy m.in przydzielanie 
  pokoi gościom na dany dzień na podstawie rezerwacji i pokoi zwalnianych,
  \item recepcjonista klucznik - m.in. zarządza kluczami, rozdziela 
  korespondencję, przyjmuje zlecenia na np. budzenie,
  \item recepcjonista ds. rezerwacji,
  \item recepcjonista kasjer - przyjmuje gotówkę, wymienia walutę,
  \item recepcjonista informator (consierge) - zajmuje się sprawami gości 
  hotelowych, np. zakup biletów na koncert.
\end{itemize}

\subsubsection{Obowiązki recepcji} 
Recepcja odpowiada za\cite[50]{OrgaDzialHot}:

\begin{itemize}
  \item przyjmowanie i realizacja zleceń na rezerwację miejsc noclegowych,
  \item przyjmowanie gości,
  \item wystawianie kart pobytu,
  \item prowadzenie ewidencji gości zgodnie z obowiązującymi zasadami,
  \item wydawanie kluczy do pokoi oraz prowadzenie właściwej ich ewidencji i 
  ich zabezpieczenie,
  \item organizowanie pomocy przy przyjazdach i wyjazdach ze szczególnym 
  uwzględnieniem opieki nad bagażem gości,
  \item udzielanie informacji gościom oraz realizowanie na ich rzecz 
  dodatkowych usług,
  \item sporządzanie codziennych grafików wykorzystania pokoi,
  \item sprawne działanie w przypadkach losowych,
  \item przyjmowanie reklamacji gości,
  \item opieka nad korespondencją,
  \item przyjmowanie mienia gości do depozytu.
\end{itemize}

\subsubsection{Zespoły recepcji}
W recepcji można wyróżnić następujące zespoły\cite[8-10]{hotel2:part1}:
\begin{enumerate}
  \item hall recepcyjny - część ogólnodostępna,
  \item lada recepcyjna,
  \item cześć wewnętrzna, usytuowana z reguły za ladą,
  \item służba parterowa.
\end{enumerate}

\section{Klasyfikacja hoteli raz jeszcze}
Wcześniej zapoznaliśmy się z klasyfikacją obiektów hotelarskich. Teraz 
chciałbym przedstawić prosty podział hoteli\footnote{obiektów hotelarskich 
rodzaju ,,hotel''}.

W pierwszym spojrzeniu na hotel, jego organizację, świadczone usługi i 
hotelowych gości można go zaklasyfikować do jednej z dwóch grup:

\begin{itemize}
  \item Hotel Turystyczny
  \item Hotel Biznesowy
\end{itemize}

Różnice będą widoczne w usługach, wystroju wewnętrznym i zewnętrznym, 
lokalizacji i nastawieniu na klienta.

Różnica będzie także w rocznym obłożeniu hoteli. Hotele turystyczne będą 
cieszyć się wysokim obłożeniem w sezonie turystycznym i zazwyczaj 
mniejszym poza okresem. Porównując obłożenia zauważymy, że hotele biznesowe 
charakteryzują się bardziej równomiernym obłożeniem w ciągu roku. 
Wyjątek stanowi okres konferencyjny, który rozpoczyna się na przełomie 
marca i kwietnia. Wówczas hotele biznesowe odnotowują zwiększone obłożenie,
 zwłaszcza te z dużą bazą konferencyjną.

Ciekawostką jest to, że pobyt w hotelu turystycznym będzie droższy w ciągu 
weekendu. Odwrotna sytuacja występuje w klasie biznesowej.

\section{Pomieszczenia hotelowe}
Hotel posiada liczne pomieszczenia. Tylko część z nich jest ogólnodostępna 
dla gości. Są to np. jednostki mieszkalne. Pomieszczenia można także 
podzielić na te, które dostępne są w określonych godzinach, a te które są 
dostępne całodobowo.

Niektóre pomieszczenia:
\begin{itemize}
\item hol
\item recepcja
\item jednostki mieszkalne
\item sala restauracyjna
\item sale konferencyjne
\item magazyny
\item zaplecze techniczne
\item biura
\item ciągi komunikacyjne
\item winda
\item korytarze
\end{itemize}

Pomieszczeń hotelowych jest o wiele więcej, ale nie ma sensu ich wszystkich 
wymieniać. Przyjrzyjmy się bliżej jednemu pomieszczeniu.

\section{Pokoje}
Najbardziej znanym pomieszczeniem hotelowym są jednostki mieszkalne, czyli 
potocznie mówiąc pokoje hotelowe. Kontynuując opis naszego przykładowego 
hotelu, będziemy mogli przedstawić rodzaje pokojów w hotelu.

\subsection{Rodzaje pokojów}
Poniżej zostały wymienione rodzaje pokojów wraz z ich angielskimi nazwami.

Rodzaje pokoi hotelowych:
\begin{itemize}
  \item pokój jednoosobowy (single room)
  \item pokój dwuosobowy (double room)
  \item pokój trzyosobowy (triple room)
  \item pokój z dostawką (room with additional bed)
  \item pokój przystosowany do obsługi gości niepełnosprawnych (room for 
  handicapped)
\end{itemize}

\section{Karta pobytu}
Karta pobytu to dokument, który otrzymuje gość po pomyślnym zakwaterowaniu. 
Karta pobytu zawiera najważniejsze informacje o dobie hotelowej, cieszy 
nocnej, warunkach bezpieczeństwa oraz usługach dodatkowych w hoteli i 
gastronomii. Może także zawierać szkic miasta z zaznaczonym hotelem.

Karta pobytu spełnia dwie funkcje:
\begin{itemize}
  \item{informacyjną}, przypomina gościom o podstawowych informacjach
  \item{zabezpieczającą} służy do identyfikacji gościa pobierającego klucz 
  lub podpisującego rachunek
\end{itemize}

\section{Rezerwacja}
Rezerwacja jest bardzo ważnym dokumentem w hotelarstwie. Jeszcze niedawno 
proces przyjmowania rezerwacji był jednym z częściej wykonywanych czynności 
przez recepcjonistę. Tam gdzie nie ma automatyzacji i możliwości dokonania 
rezerwacji przez internet nadal tak jest. Tam gdzie jest to proces manualny, 
tam też istnieje możliwość pomyłki. Prawidłowe dokonanie rezerwacji jest 
sprawą kluczową zarówno dla gościa jak i dla hotelu. Rezerwacja może bowiem 
stanowić źródło wielu dodatkowych informacji przydatnych przy organizacji 
usług hotelarskich i działań marketingowych.

W następnych podrozdziałach przyjrzymy się bliżej rezerwacji, gdyż jest to 
kluczowy element. Zapoznamy się w jaki sposób można jej dokonać, jakie są 
typy rezerwacji, jakie informacje zawiera i wiele dowiemy się o wielu innych 
rzeczach związanych z rezerwacją.

\subsection{Co oznacza rezerwacja?}
Rezerwacja, która jest niepotwierdzona od strony prawnej nie wiele znaczy. 
Między przyjęciem rezerwacji, a jej potwierdzeniem istnieje wiele wariantów, 
które mogą się wydarzyć w zależności od hotelu, ale z tym zapoznamy się 
później. 

Przyjęcie i potwierdzenie rezerwacji jest natomiast formą zawarcia umowy 
między gościem, a hotelem do realizacji podstawowej usługi hotelarskiej, 
czyli gościnności za odpłatnością. Aby zatem przyjąć rezerwację trzeba 
wiedzieć o obłożeniu hotelu, planowanych imprezach, planowanych remontach, 
tak aby móc zrealizować usługę w określonym terminie. Rezerwacja w ogólności 
jest dokonywana na typ pokóju, a nie konkretny pokój aczkolwiek może być 
inaczej. Wynika z tego, że trzeba wiedzieć jakie typy będą dostępne w 
określonym terminie.

\subsection{Sposoby przeprowadzania rezerwacji}
Sposobów dokonania rezerwacji jest kilka. Oto one:
\begin{itemize}
  \item dokonanie rezerwacji bezpośrednio w hotelu,
  \item dokonanie rezerwacji za pomocą faksu,
  \item dokonanie rezerwacji za pomocą internetu,
  \begin{itemize}
     \item bezpośrednio na stronie hotelu - uzupełniając formularz na 
     stronie hotelu
     \item pośrednio poprzez system klasy CRS \footnote{przykładowo www.
     booking.com/index.pl.html} \ref{system-crs}
  \end{itemize}
  \item dokonanie rezerwacji poprzez rozmowę telefoniczną - podając dane 
  pracownikowi recepcji. Przez całą procedurę przeprowadzi nas pracownik w 
  recepcji.\footnote{W sieci hoteli Marriott wszystkie dane powtarzane są 
  dwa razy w celu sprawdzenia ich poprawności przez obie strony},
  \item dokonanie rezerwacji poprzez biuro usług pośredniczących - które  
  może korzystać z wcześniej wymienionych sposobów lub poprzez system klasy 
  CRS \ref{system-crs}. Istnieje jeszcze jedna możliwość, którą poznamy przy 
  okazji rezerwacji grupowych w \ref{rezerwacja-grupowa}
\end{itemize}

\subsection{Rodzaje rezerwacji}
W celu zabezpieczenia hotelu przed stratami z tytułu anulowanych rezerwacji, 
a tym samym niewykorzystanych pokoi wprowadzono trzy rodzaje rezerwacji:

\begin{itemize}
  \item wstępną,
  \item gwarantowaną,
  \item niegwaratnowaną.
\end{itemize}

To czy wszystkie rodzaje rezerwacji są dostępne zależy od hotelu. 

\subsubsection{Rezerwacja wstępna}
Dokonujemy rezerwacji i otrzymujemy termin, do którego musimy potwierdzić 
rezerwację. W przypadku braku potwierdzenia rezerwacja zostaje anulowana. 
Żadna ze stron nie ponosi konsekwencji.

Jaki to będzie termin, zależeć będzie od hotelu. W przypadkach, gdzie 
głównie używana jest rezerwacja gwarantowana, może być tak, że terminem 
będzie godzina popołudniowa tego samego dnia, w którym dokonujemy rezerwację.
 Zatem ma to sens, jeśli już jesteśmy w drodze do hotelu i chcemy zrobić 
 rezerwację bez podawania numeru karty kredytowej.

\subsubsection{Rezerwacja gwarantowana}
Jest to typowy rodzaj rezerwacji. Aby dokonać rezerwacji potrzebna jest 
pewna forma gwarancji, a dokładnie pewne zabezpieczenie finansowe. Jest to 
najbezpieczniejszy rodzaj rezerwacji dla hotelu.

Forma zabezpieczenia może być różna. Może to być przedpłata w określonym 
terminie. Może to być także obciążenie karty kredytowej. Karta kredytowa to 
najczęściej spotykany sposób w przypadku dużych sieci hotelarskich. W naszym 
rodzimym przypadku spotkać się można z przedpłatą bankową.

Rezerwacja gwarantowana to sposób obrony hotelu przed stratami finansowymi. 
Jeśli gość zrezgynuje z rezerwacji to w zależności od terminu albo straci 
całe zabezpieczenie, albo część. Szczegóły zależą od zakładu hotelarskiego, 
ale zazwyczaj trzeba liczyć się z pewną karą.

Jeśli gość nie pojawi się planowanego dnia przyjazdu, to rezerwacja zostaje 
utrzymywana do dnia następnego, a gość zostaje obciążony kosztami pokoju za 
jedną noc, nie wliczając śniadania. Dalsza procedura zależy od hotelu, ale 
prawdopodobnie pracownik skontaktuje się z gościem, aby wyjaśnić zaistniałą 
sytuację. Koniec końców największy zysk jest z zadowolonego gościa 
hotelowego, a nie kar finansowych, które to zadowoleniu szkodzą.

Warto dodać, że metoda dokonania gwarancji nie ma związku z metodą płatności 
za pobyt.

\subsubsection{Rezerwacja niegwarantowana}
W odróżnieniu od rezerwacji gwarantowanej, dla tego typu rezerwacji nie ma 
konieczności zabezpieczenia w postaci depozytu. W przypadku nie 
wykorzystania rezerwacji w określonym terminie umowa przestaje obowiązywać, 
a jej rozwiązanie nie pociąga ze sobą żadnych skutków finansowych. Całe 
ryzyko ponosi zatem hotel. Łatwo sobie wyobrazić złe działania konkurencji, 
które mogą wykorzystać ten sposób rezerwacji.

\subsection{Pojedyncza i grupowa rezerwacja}
Kolejnym podziałem rezerwacji jest podział na rezerwację indywidualną oraz 
grupową. 

\subsubsection{Rezerwacja indywidualna}
Rezerwacja indywidualna to rezerwacja dla klienta indywidualnego, czyli taka 
jaką sami możemy zamówić w hotelu. Nie oznacza to jednak, że pobyt musimy 
spędzić samemu. Oznacza to, że rezerwacji dokonuje zwykła osoba. Dla tego 
typu rezerwacji ceny za pokój będą najwyższe.
Rezerwacja indywidualna może dotyczyć więcej niż jednego pokoju.

Zamiast samemu dokonywać rezerwacji możemy posłużyć się biurem pośrednictwa 
rezerwacji. Są to m.in biura turystyczne, biura linii lotniczych. Możliwe, 
że uda nam się wtedy zapłacić mniej, a dlaczego tak jest dowiemy się w 
\ref{biuro-turystyczne}

\subsubsection{Rezerwacja grupowa}
\label{rezerwacja-grupowa}
Rezerwacja grupowa to rezerwacja dla klienta grupowego. Rezerwacji grupowych 
dokonuje wiele różnych instytucji. Są to m.in biura turystyczne, korporacje, 
stowarzyszenia, agencje podróży. Za wszystkie formalności związane z 
dokonaniem rezerwacji odpowiedzialny będzie pośrednik wynajęty przez grupę, 
bądź organizatora pobytu dla grupy. Grupa oznacza wiele osób - rezerwacja 
grupowa dotyczyć będzie zatem wielu pokojów. 

Opłata za rezerwację grupową zależeć będzie od sposobu jej organizacji. W 
ogólnym przypadku pobyt opłacony będzie poprzez organizatora grupy. 
Natomiast to czy dodatkowe usługi\footnote{usługi tworzyszące i fakultatywne}
 również będą wliczone w cenę zależy od konkretnego przypadku. Istnieją trzy 
 możliwości:

\begin{itemize}
   \item organizator płaci za dodatkowe usługi,
   \item organizator nie płaci za dodatkowe usługi. Każdy członek grupy 
   płaci za korzystanie z usług,
   \item organizator płaci za dodatkowe usługi do pewnej kwoty. Nadwyżkę 
   opłaca członek grupy.
\end{itemize}

Ceny za pobyt będą atrakcyjniejsze w przypadku rezerwacji grupowych.

\subsubsection{Biuro turystyczne}
\label{biuro-turystyczne}
Biuro turystyczne jest jednym z podmiotów, które dokonują rezerwacji 
grupowych. Biura turystyczne posiadają także oferty dla klientów 
indywidualnych, w których koszta pobytu są niższe. Dlaczego tak jest?

Powody mogą być conajmniej dwa. Pierwsza możliwość jest taka, że biuro 
podróży nawiązuje z hotelem umowę, w której rezerwuje określoną liczbę 
pokoi wraz z określonymi usługami przy ustalonych warunkach cenowych - 
które będą korzystniejsze niż dla klienta indywidualnego. Najważniejszym dla 
biura jest termin, do którego musi powiadomić hotel o faktycznym 
wykorzystaniu miejsc. Jeśli w wyznaczonym terminie powiadomi hotel i 
zmniejszy ilość pokoi to obciążone zostanie tylko za wykorzystaną ilość, a 
reszta pokoi wróci do wolnej puli. Przy takiej umowie hotel zobowiązuje się, 
że określona ilość pokoi będzie dostępna i że nie będą one przedmiotem 
podobnej umowy z innym biurem turystycznym bądź inną instytucją.

Druga możliwość to tzw. \emph{allotment}. Biuro podróży wykupuje określoną 
ilość 
miejsc za bardzo atrakcyjną cenę. Biuro podróży nie może zrezygnować z 
wykupionych miejsc. Hotel nie może zawierać na te miejsca innych umów.

Tymi sposobami biuro podróży może pozwolić sobie zaoferować niższe ceny. 
Jeśli skorzystamy z oferty otrzymamy Voucher.

\begin{defn}{Voucher} 
to dokument indywidualny podobny do rezerwacji, który uprawnia beneficjenta 
do skorzystania z usługi.
\end{defn}

\subsection{Informacje na rezerwacji}
Przy dokonywaniu rezerwacji możemy być pytani o wiele różnych informacji. 
Istnieje w miarę standardowa porcja informacji, którą zawsze trzeba podać. 
To ile szczegółowych informacji będziemy mogli podać zależeć będzie od hotelu
. Bardzo szczegółową informacją są np. preferencje dotyczące rodzaju 
poduszki, bądź umiejscowienia pokoju blisko windy.

Podstawowe informacje, które podajemy przy dokonaniu rezerwacji to:

\begin{itemize}
  \item data przyjazdu,
  \item data wyjazdu,
  \item rodzaj pokoju,
  \item dane rezerwującego (klienta)
  \begin{itemize}
    \item tytuł,
    \item imię,
    \item nazwisko, 
    \item adres, 
    \item dane kontaktowe
  \end{itemize} 
  \item dane dla kogo jest rezerwacja
  \begin{itemize}
    \item tytuł,
    \item imię
    \item nazwisko
    \item osoby dodatkowe
  \end{itemize}
  \item metodę płatności
\end{itemize}

Niektóre z dodatkowych informacji, o które pytać będą hotele o wysokim 
standardzie:

\begin{itemize}
  \item czy pokój dla palących czy niepalących,
  \item umiejscowienie pokoju,
  \begin{itemize}
    \item niskie piętro,
    \item wysokie piętro,
    \item blisko windy,
  \end{itemize}
  \item widok,
  \item rodzaj poduszki,
  \item ile osób dorosłych,
  \item ile dzieci,
  \item informacje dodatkowe dla recepcji - specjalne życzenia bądź uwagi,
  \item preferencje żywieniowe.
\end{itemize}


\subsubsection{Potwierdzenie rezerwacji}
Potwierdzenie rezerwacji potwierdza dokonanie rezerwacji i tym samym jest 
zawiązaniem umowy między gościem, a hotelem. Potwierdzenie rezerwacji może 
być przekazane drogą elektroniczną bądź tradycyjną. Potwierdzenie rezerwacji 
pozwala na sprawdzenie poprawności danych. Potwierdzenie rezerwacji stwarza 
możliwość reklamy innych usług zakładu hotelarskiego. 

Sam dokument z potwierdzeniem zawierać będzie:
\begin{itemize}
  \item numer rezerwacji
  \item datę zgłoszenia rezerwacji
  \item adres hotelu
  \item informacje podane przy dokonaniu rezerwacji
  \item podsumowanie kosztów za pokój
\end{itemize}

Może zawierać także numer rachuneku bankowego, na który trzeba wpłacić 
zaliczkę wraz z terminem wpłaty.

\subsection{Odwoływanie rezerwacji}
Polityka odwołania rezerwacji powinna być dostępna podczas dokonywania 
rezerwacji. Zazwyczaj jest to możliwe i nie wiąże się z karami finansowymi, 
jeśli rezerwacja zostanie odwołana przed określonym terminem. 

Zazwyczaj, aby odwołać rezerwację należy się skontaktować telefonicznie z 
hotelem. Jeśli robimy to w dniu planowanego przyjazdu, to wiele 
zależeć będzie od pracownika hotelu. Warto pamiętać, że pracownik hotelu ma 
dostęp do informacji o lotach i może łatwo zweryfikować informację o 
spóźnionym przylocie. Ważne jest tutaj odpowiednie przeszkolenie pracowników 
i ich doświadczenie oraz indywidualne podejście do sprawy.

Jeśli nie pojawimy się w hotelu w dni przyjazdu to typowym mechanizmem jest 
to, że nasza rezerwacja oznaczona będzie jako tzw. \emph{No show}. 

"TODO dodać jakoś lepiej to źródło"
Źródło definicji: http://abchotelu.pl/
\begin{defn}{No show}
nie pojawienie się gościa w hotelu nie poprzedzone anulowaniem rezerwacji. 
Wiąże się z poniesieniem przez Gościa ustalonych kosztów w zależności od 
rodzaju rezerwacji. 
\end{defn}

Dalsza procedura to kontakt z rezerwującym następnego dnia, wyjaśnienie 
sytuacji, a w ostateczności obarczenie karą\emph{np. wartość zabezpieczenia 
+ koszt jednej nocy}. Warto pamiętać o tym, że karanie klientów nie przynosi 
długoterminowych zysków dlatego wspólny dialog z klientem i porozumienie 
jest właściwą drogą rozwiązywania tego typu problemów.

\subsection{Relokacja rezerwacji}
Relokacja rezerwacji polega na przesunięciu w czasie daty przyjazdu lub/i 
wyjazdu. Czy odbędzie się to bez dodatkowych kosztów zależeć będzie od 
hotelu oraz terminu, w którym dokonywać będziemy relokacji. Możliwe jest 
również, że cena pokoju może być wyższa w nowym terminie i zapłacimy wtedy 
więcej. Warunkiem koniecznym do relokacji jest to, aby w nowym terminie 
pokój był dostępny. Jeśli nie byłby dostępny to prawdopodobnie pracownik 
hotelu zaoferuje nam inny tańszy bądź droższy pokój, który będzie dostępny w 
tym terminie.

\subsection{Stan rezerwacji}
Najważniejsze stany rezerwacji to:
\begin{itemize}
  \item nie potwierdzona - (ang. requested) żądanie rezerwacji zostało przyjęte, ale pobyt nie został jeszcze potwierdzony,
  \item potwierdzona - (ang. reserved) pobyt został potwierdzony,
  \item odwołana - (ang. cancelled) rezerwacja została odwołana,
  \item przyjazd - (ang. check-in) dzisiejszy dzień jest dniem przyjazdu gościa,
  \item pobyt - (ang. in-house) gość odbywa pobyt w hotelu,
  \item wyjazd - (ang. check-out) dziejszy dzień jest dniem wyjazdu,
  \item no show - gość nie pojawił się w dniu przyjazdu.
\end{itemize}

\section{Rodzaje klientów}

\section{Cennik}

\subsection{Najwyższa stawka}

\subsection{Stawka okresowa}

\subsection{Cena wolna}

\section{Pakiety}

TODO opis np. weekend + coś tam dla dwójki w walentynki

\section{OpenTravel}

\section{Systemy CRS}
\label{system-crs}
Centralny System Rezerwacji \footnote{tłumaczenie z ,,Computer reservation 
systems'' nazywane także ,,Central reservation systems''} CRS to systemy 
informatyczne służące do przechowywania, odzyskiwania danych oraz do 
przeprowadzania transakcji związanych z podróżą. Pozwalają na rezerwację 
oraz sprzedaż np. biletów lotniczych, pokojów hotelowych, samochodów do 
wynajęcia. Klient takiego systemu ma dostęp do wielu linii lotniczych, 
hoteli i innych przedsiębiorstw związanych z brażną podróżniczą, które są 
zintegrowane z systemem.

Do największych należą m.in.:
\begin{itemize}
\item Abacus
\item AccelAero
\item Amadeus
\end{itemize}

Podane powyżej systemy mają zasięg globalny. System CRS, który ma zasięg 
globalny nazywa się także systemem GDS - Globalny System Dystrybucji\footnote
{tłumaczenie z ,,Global Distribution Systems''} Rozwój technologiczny 
sprawił, że powstało także wiele podobnych rozwiązań na mniejszą skalę 
stosowanych przez mniejsze sieci hotelowe i inne przedsiębiorstwa branży 
podróżniczej. Uprościł się także dostęp do tych systemów dla klienta 
indywidualnego bez ograniczeń terytorialnych. Może on zawierać transakcję z 
dowolnego miejsca na świecie.

Znaczna część dużych hoteli podłączonych jest pod jeden lub więcej systemów 
klasy CRS.
Za pośrednictwem systemów CRS dokonuje się ponad jedną czwartą wszystkich 
rezerwacji w sektorze hotelarstwa.
TODO dodać źródło tej statystyki lub wywalić ten fakt

\section{Systemy hotelarskie}

\subsection{Porównanie}
TODO może jakieś porównanie systemów hotelarskich

% TODO ten rozdział powinien być za rozdziałem o rezerwacji i jej typach %%
\section{Rodzaje klientów}
\label{rodzaje-klientow}
Goście hotelowi podlegają podziałowi ze względu na w jaki sposób odbywają 
swój pobyt, jak za niego płacą i jakie usługi są im świadczone.

Podział klientów wygląda następująco:

\begin{itemize}
  \item klient indywidualny,
  \item klient grupowy,
  \item klient z biura turystycznego,
  \item klient korporacyjny,
  \item klient stały.
\end{itemize}

\subsubsection{Klient indywidualny}

TODO TODO TODO

%%% OLD BELOW %%%%

\section{Branża hotelarska}

\subsection{Wstęp}

Branża hotelarska jest bardzo różnorodna, począwszy od ogromnych sieci
hotelarskich o zasięgu światowym, które obsługują miliony klientów dziennie i zatrudniają
 jeszcze więcej osób przez średniej wielkości hotele, które działają
 samodzielnie, aż po małe, rodzinne biznesy oferujące swoje usługi
 w turystycznych miejscowościach.
 
  Wszystkie mają jedną wspólną cechę,
 świadczą usługi noclegowe dla swoich klientów. Jest to podstawowy element
 działalności hotelarskiej. Cała reszta usług ma sprawiać, aby klient był
 bardziej zadowolony z pobytu, ponieważ to on jest tutaj najważniejszy.
 
  Dochodzimy tutaj do prostego wniosku, że znakomita
 większość spraw związanych z prowadzeniem hotelu jest taka sama jak dla innych
 biznesów i tyczą się ich te same problemy. W innych biznesach, tam gdzie mamy
 zamówienia, tutaj występują rezerwacje, a towar jest pobytem w hotelu. Istnieje
 cały szereg analogii i dopiero na tym poziomie można zobaczyć czym tak naprawdę
 różni się prowadzenie hotelu od chociażby sklepu wysyłkowego.
 
 Istnieje jednak pewna subtelna różnica pomiędzy branżą hotelarską, a innymi
 branżami. Mianowicie w tym segmencie biznesu jesteśmy bardzo, ale to bardzo
 zależni od poziomu zadowolenia klienta i zależy nam na tym, aby poziom ten był
 jak najwyższy. Jeden nieusatysfakcjonowany klient to strata kilkunastu
 innych potencjalnych klientów, którym ów klient odradzi pobyt w naszym hotelu.
 Jest to szczególnie dotkliwe dla tych mniejszych i średnich hoteli, gdzie nie
 można pozwolić sobie na taką stratę.

\subsection{Klasyfikacja}
W pierwszym spojrzeniu na hotel, jego organizację i hotelowych gości można go
zaklasyfikować do jednej z dwóch kategorii:
\begin{itemize}
  \item Turystyczny
  \item Biznesowy 
\end{itemize}

Hotele, które należą do jednego albo drugiego profilu biznesowego różnią się pod
bardzo wieloma aspektami. Najważniejsze z nich to lokalizacja, wystrój
zarówno wewnętrzny jak i zewnętrzny, rodzaj klientów oraz świadczone usługi
w hotelu. Typowe również jest to, że dla hoteli turystycznych weekendy są zazwyczaj
 droższe niż pobyty w środku tygodnia. Sytuacja jest zupełnie odwrotna dla grupy
 biznesowej. Charakterystycznym dla hoteli turystycznych jest to, że obłożenie
 hotelu jest wysokie zazwyczaj tylko w tzw. sezonie turystycznym, a przez resztę
 roku utrzymuje się na niższym poziomie.
 
 W klasie hoteli o profilu biznesowym średnie obłożenie będzie zazwyczaj wyższe
 niż dla tych z grupy turystycznej, ale na pewno nie jest to reguła, która się
  sprawdza zawsze. Tutaj również istnieje sezon, w którym obłożenie hotelu
  osiąga swój szczyt. Pomijając specjalne okoliczności takie jak EURO2012, takim
  sezonem jest sezon konferencyjny, który zaczyna się na przełomie marca i kwietnia.
  
  Dla hoteli z profilu biznesowego następuje 
 jeszcze jeden podział ze względu na politykę wynajmowania albo łóżek albo pokoi.
   
 Wytłumaczenie różnic w politykach:
 
 W jednym hotelu możemy wynająć pokój np.
 3 osobowy i cena pozostanie taka sama jeśli będziemy tam sami albo w dwie osoby albo w pięć jeśli jest to hotel,
  który działa zgodnie z polityką wynajmowania pokoi. Inna sytuacja nastąpi w hotelach, które wynajmują łóżka,
   cena będzie się różniła dość znacznie. Takie hotele również
   zazwyczaj nie godzą się na to aby w pokoju nocowało więcej osób niż liczba łóżek. 
   Hotele klasy turystycznej zazwyczaj prowadzą politykę wynajmowania łóżek, a hotele klasy biznesowej wynajmowania pokoi.

\subsection{Rezerwacja}
Elementem wspólnym dla każdego hotelu jest rezerwacja.
Rezerwacja to dokument, który zapowiada pewne zdarzenie
z przyszłości jakim jest pobyt w hotelu w określonych dniach,
pokoju, warunkach i osobach, które ten pobyt będą odbywać.

Niestety nie ma utartego schematu jednej rezerwacji i tego jakie informacje
powinna zawierać. Można wyszczególnić pewien zbiór informacji, który zawiera się
na każdej rezerwacji, a przynajmniej powinien się zawierać. Jednak istnieją
hotele, które pozwalają na wprowadzanie wielu dodatkowych informacji do
rezerwacji. Zazwyczaj są to hotele wysokiej klasy, a informacje które możemy
podać odnoszą się np. do typu poduszki jaki chcemy, rodzaju łóżka lub innych
upodobań. Więcej informacji o tym co może zawierać informacji znajdziesz w
sekcji \ref{informacje_na_rezerwacji}


\subsubsection{Gwarantowana rezerwacja}
Rezerwacja zazwyczaj wymaga pewnej gwarancji ze strony rezerwującego, ponieważ
hotel zobowiązuje się, aby pokój, na który prowadzona jest rezerwacja
był w danym czasie dostępny. Jeśli kto inny chciałby wtedy dokonać rezerwacji 
zakładając, że był to ostatni wolny pokój to wtedy stracimy klienta jeśli 
rezerwujący się wycofa albo nie przyjdzie. Potrzebny jest zatem pewien  
mechanizm zabezpieczający hotel przez potencjalnymi stratami. 

Typowy mechanizm gwarantowanej rezerwacji dla klientów indywidualnych przy dokonywaniu 
jej online polega na podaniu numeru karty kredytowej, która w przypadku 
nieodwołania rezerwacji przed określonym terminem przez hotel zostanie obciążona pewną karą 
również określoną przez hotel. Dla klientów indywidualnych niektóre hotele 
umożliwiają dokonywanie rezerwacji niegwarantowanej. Polega to na tym, 
że można dokonać rezerwacji z datą przyjazdu na dzisiaj bez podawania 
numeru karty, ale jest ona trzymana do określonej godziny, 
zazwyczaj 16 czasu lokalnego, po której to już nie mamy pewności 
czy pokój nie zostanie wynajęty komuś innemu. Możliwość dokonania
niegwarantowanej rezerwacji jest świadczona tylko przez niektóre hotele.
Zazwyczaj każdy woli być zabezpieczony. 

Niegwarantowana rezerwacja to dość wygodne rozwiązanie szczególnie w krajach jak
Polska, gdzie posiadanie karty kredytowej nie jest jeszcze tak popularne. Warto zaznaczyć, że sposób gwarancji rezerwacji 
nie ma nic wspólnego z metodą płatności za pobyt. Przykładowo gwarantować
rezerwacje można kartą kredytową, a płacić gotówką albo przelewem albo innym
środkiem płatności akceptowanym przez hotel.


\subsubsection{Informacje na rezerwacji}
\label{informacje_na_rezerwacji}
Rezerwacja przechowuje dość sporo informacji. Standardowa porcja informacji, 
które są wymagane aby dokonać rezerwacji jest następująca:
\begin{itemize}
  \item Numer rezerwacji,
  \item data przyjazdu,
  \item data wyjazdu,
  \item rodzaj pokoju,
  \item dane rezerwującego
  \begin{itemize}
    \item imię,
    \item nazwisko, 
    \item adres, 
    \item kontakt jak e-mail, numer telefonu.
  \end{itemize} 
  \item kto płaci
\end{itemize}

Oprócz takich podstawowych informacji bardzo często możemy podać także wiele 
innych informacji. Często umożliwiają to hotele o wysokim standardzie. Niektóre
z możliwych informacji to np. preferencje:
\begin{itemize}
  \item czy pokój dla palących czy niepalących,
  \item umiejscowienie pokoju,
  \begin{itemize}
    \item niskie piętro,
    \item wysokie piętro,
    \item blisko windy,
  \end{itemize}
  \item widok,
  \item rodzaj poduszki.
\end{itemize}

Podać można także takie informacje jak:
\begin{itemize}
  \item kod promocyjny - często hotele biznesowe, a szczególnie duże sieci
  hotelarskie o światowym zasięgu oferują programy lojalnościowe, w których
  zbiera się punkty, które później można wymienić np. na voucher na darmowy
  pobyt w hotelu. Kod promocyjny odnosi się właśnie do tych nagród,
  \item przynależność do grupy,
  \item przynależność do korporacji,
  \item przynależność do agenta biura podróżniczego,
  \item planowana godzina przyjazdu,
  \item informacje dodatkowe dla recepcji jak specjalne życzenia bądź uwagi
\end{itemize}

Inne informacje zależą tylko od inwencji własnej osoby odpowiedzialnej za hotel,
która będzie chciała zrobić tak żeby klienci czuli się jak najlepiej. Część z nich 
taka jak odpowiedni kod grupy lub korporacji ma związek z typem klientów hotelu.
Typ klientów został opisany w sekcji \ref{rodzaje_klientow}

\subsubsection{Sposób dokonywania rezerwacji}
Rezerwacji można dokonywać:
\begin{itemize}
  \item online – uzupełniając formularz na stronie hotelu
  \item telefonicznie – podając dane pracownikowi recepcji. Wtedy przez 
  całą procedurę przeprowadzi nas pracownik recepcji. Wszystkie dane powtarzane
  są dwa razy w celu sprawdzenia ich poprawności przez obie strony.
  \item osobiście – jeśli chcemy dokonać osobiście rezerwacji w hotelu w
  recepcji to możemy to zrobić, po czym dostajemy albo klucz do 
  pokoju – jeśli chcieliśmy mieć pokój na teraz albo potwierdzenie rezerwacji, 
  które jest dokumentem zawierającym podstawowe dane rezerwacji oraz 
  numer rezerwacji
\end{itemize}

\subsubsection{Odwołanie rezerwacji}
Odwołanie rezerwacji jest zazwyczaj możliwe i nie wiąże się z karą jeśli 
rezerwacja zostanie odwołana przed określonym terminem. Zazwyczaj, ponieważ
niektóre hotele całkowicie zabraniają odwoływania rezerwacji zatem zawsze wiąże
się to z utrzymaniem karą. Przykładem są, niektóre hotele w sieci Marriott.
Te hotele, które umożliwiają odwołanie rezerwacji wymuszają pewne
zasady na których to odwołanie się odbywa zazwyczaj jest to określona godzina,
do której możemy odwołać rezerwację. Przykładowy termin to godzina 16 czasu
lokalnego hotelu w planowanym dniu przyjazdu. Może to być także zupełnie inny termin arbitralnie wybrany przez osobę za to odpowiedzialna. 
  Po określonym terminie odwołanie jest możliwe jeśli hotel ma taką politykę. 
  Zazwyczaj odwołanie rezerwacji przeprowadza się telefonicznie dlatego także to w gestii 
  pracownika hotelu jest ustalenie co zrobić, czy odwołać rezerwację bez żadnej kary nawet po terminie
   czy jeśli klient coś kręci to nie dać się nabrać i odpowiednio zareagować. 
  Ważne jest tutaj odpowiednie przeszkolenie pracowników i ich doświadczenie.
  
  Należy pamiętać o tym, że jeśli mówimy, że nie możemy przyjechać w danym dniu,
  ponieważ lot nam się opóźnił to taka informacja jest łatwa do sprawdzenia
  przez pracownika hotelu i jeśli pracownik ma podejrzenie, że coś jest nie tak
  to właśnie zostanie to sprawdzone.

Jeśli nie pojawimy się w hotelu w dniu rezerwacji to typowym mechanizmem jest
to, że w następnym dniu rezerwacje które nie zostały zrealizowane oznaczane są
jako NO SHOW. Dalsza procedura zależy od hotelu albo obarczą karą tego kto
gwarantował rezerwację albo będą próbowali skontaktować się z rezerwującym. Nie
ma tutaj mechanizmu uniwersalnego. Należy jednak pamiętać, że karanie klienta
jest rozwiązaniem, które przynosi korzyści tylko krótkoterminowo, a hotel
powinien być zainteresowany zyskami długoterminowymi dlatego też warto aby
istniała procedura opierająca się na wyjaśnienie przyczyny, dialog z klientem i
wspólne porozumienie.

\subsubsection{Relokacja rezerwacji}
Relokacja rezerwacji polega na przesunięciu w czasie daty 
przyjazdu lub/i wyjazdu. Jest to możliwe do zrobienia. Warto jednak pamiętać o
tym że o ile sama relokacja nic nie kosztuje to cena rezerwacji może 
wzrosnąć, jeśli przesuwamy ją na termin, w którym cena pobytu jest po prostu 
większa. Prosta zasada to nowy termin, nowa wycena. Relokacja jest możliwa 
wtedy gdy mamy wolne pokoje w terminie, na który klient chce przesunąć
rezerwację.

\subsubsection{Stan rezerwacji}
Rezerwacja znajduje się w kilku możliwych stanach. Czytelników wrażliwych na
anglicyzmy zapraszam do następnej sekcji. Osobiście nie podejmuje się tłumaczeń
poszczególnych stanów, a jedynie wyjaśniam co znaczą. Zaletą jest to, że
praktycznie wszędzie na świecie spotkamy te same angielskie nazwy. Pierwsze dwie
stany są trochę bliskie wewnętrznej organizacji systemu, reszta jest już istotna
z punktu widzenia człowieka.

\begin{description}
\item[Request denied] żądanie rezerwacji zostało odrzucone
\item[Requested] żądanie rezerwacji zostało przyjęte, ale pobyt nie został
jeszcze zarezerwowany
\item[Reserved] pobyt został zarezerwowany
\item[Cancelled] rezerwacja została odwołana przez klienta
\item[In-house] klient zameldował się do hotelu. Od tego momentu zaczyna się
pobyt
\item[No-show] klient na którego była rezerwacja nie wmeldował się w dniu
przyjazdu
\item[Checked out] gość hotelowy wymeldował się z hotelu. Koniec pobytu.
\item[Waitlisted] rezerwacja oczekuje na potwierdzenie
\end{description}

Można by jeszcze wskazać jeden stan, a mianowicie stan \emph{Check-in}.
Rezerwacja znajduje się w tym stanie, gdy obecny dzień jest dniem przyjazdu
klienta.

\subsubsection{Typowy przebieg}
Typowy przebieg wygląda następująco. Składamy rezerwację przez internet i ma
wtedy ona stan \emph{Requested}, a wkrótce potem \emph{Reserved}. Następnie
nadchodzi dzień przyjazdu i rezerwacja ma stan \emph{Check-in}. Gdy przyjeżdżamy
do hotelu i zgłaszamy się w recepcji recepcjonista/tka przyjmuje nas, a
rezerwacja zmienia swój stan na \emph{In-house}. Odbywamy pobyt i cieszymy się z
wypoczynku, aż przychodzi dzień wyjazdu. Wymeldowujemy się z hotelu, a
rezerwacja zmienia swój stan na \emph{Checked out}.

\subsection{Rodzaje klientów}
\label{rodzaje_klientow}
Gości hotelowych można podzielić na parę kategorii. Przynależność do danej
kategorii ma wpływ przede wszystkim na cenę pobytu jak również na inne 
dodatkowe usługi oferowane przez hotel. 

Podział wygląda następująco:
\begin{itemize}
  \item klient indywidualny,
  \item klient korporacyjny,
  \item klient grupowy,
  \item klient związany z agentem biura podróżniczego,
  \item klient stały.
\end{itemize}

\subsubsection{Klient indywidualny}
Klient indywidualny jest to po prostu zwykła osoba z ulicy, która zazwyczaj
płaci najwyższą stawkę za pobyt i inne usługi dodatkowe jakie oferuje hotel.

\subsubsection{Klient korporacyjny}
Klient, który należy do korporacji, która ma z hotelem podpisaną umowę.
Zazwyczaj wygląda to tak, że korporacja zobowiązuje się do wykorzystania 
pewnej liczby pobytów w roku co powoduje iż cena jest odpowiednio niższa.
Wszystkie szczegóły reguluje umowa.

\subsubsection{Klient grupowy}
W hotelach czasem organizowane są konferencje, bale bądź wystawy. Uczestnik
takiego wydarzenia należy do grupy.

Zdarza się także, że na czas pobytu grupy w hotelu w recepcji ustawiane 
jest dodatkowe stanowisko, które obsługuje tylko i wyłącznie członków danej 
grupy. Taka sytuacja zazwyczaj ma miejsce w większych hotelach.

Ceny dla takich grupowych klientów, podobnie jak z klientami korporacyjnymi,
reguluje umowa pomiędzy hotelem, a organizatorem danego wydarzenia.

\subsubsection{Klient związany z agentem biura podróżniczego}
Przy wyszukiwaniu miejsca na spędzenie wakacji udajemy się zazwyczaj do biura
podróżniczego, które oferuje nam kompleksowo zaplanowane wakacje.

Między innymi elementami układanki znajduje się hotel, w którym odbędziemy pobyt
i właśnie taki klient jest klasyfikowany do tej grupy.

Również w tym przypadku występuje umowa pomiędzy agencją turystyczną, a hotelem.

\subsubsection{Klient stały}
Istnieją tacy klienci którzy po prostu mieszkają w hotelach. Przykładem może być
Pan Krauze i Hotel Marriott.  

Taki klient ma gwarancje tego, że zawsze dostaną
swój ulubiony pokój więc zazwyczaj jest on dla nich trzymany wolny. Klient stały
ma szereg udogodnień zarówno w usługach jak i płatnościach np. przekładanie
płatności lub łączenie ich za wiele pobytów.

Tutaj istnieje ogromna dowolność w obsługiwaniu takiego klienta i nie należy
tego ograniczać żadnymi ramami.

\subsubsection{Grupy w kontekście rezerwacji}
Członkowie grupy lub klienci korporacyjni mają wszystko albo opłacone 
albo płacą sami za siebie, wtedy jest to cenowo nadal mniej niż dla klientów 
indywidualnych. W przypadku gdy wszystko jest opłacone po prostu
wystawiana jest faktura na odpowiednią firmę, z którą związana jest dana
grupa. Wszystko zależy od tego jak sporządzona zostanie umowa pomiędzy hotelem,
a innym podmiotem gospodarczym.

Każdy uczestnik grupy ma swoją własną rezerwację. To, że należy do grupy jest 
odnotowane w informacjach zawartych w rezerwacji. To samo tyczy się klientów 
należących do innej kategorii. Rezerwacja jest wystawiana na konkretne osoby, 
a nie firmy z którymi podpisana jest umowa.

\subsection{Długość pobytów}
Wiadomym jest, że istnieje dowolność w długości pobytów, choć czasem mogą być
pewne różnice w cenie za noc jeśli spędzimy jedną noc lub trzy pod rząd.

To co jednak nie jest tak bardzo rozpowszechnione to to, że w hotelach,
zazwyczaj w tych z klasy biznesowej, możliwe jest odbycie krótkoterminowego
pobytu.
\subsubsection{Krótkoterminowy pobyt}
Jeśli potrzebujemy się tylko odświeżyć albo spędzić kilka godzin w klientem na
rozmowie albo wystąpił inny realny scenariusz to warto skorzystać z opcji jaką
jest pobyt do godziny po południowej.
Przykładowo możemy wynająć pokój w taki sposób, że musimy go opuścić np. do
godziny 16 i wtedy nie zapłacimy standardowo za noc, tylko odpowiednio mniej.
Nie jest to przelicznik godzinowy. Po prostu można odbyć pobyt do południa i
koszt takiego pobytu będzie na pewno mniejszy niż byśmy wynajęli pokój na noc.

\subsection{Konferencje}
Wydawać by się mogło, że głównym źródłem dochodu dla hotelu jest oferowanie
pokoi. Jednak często jest tak, że spora część dochodu dla hotelu ma źródło z
przyjmowania konferencji. Oczywiście nadal w grę wchodzą pokoje dla uczestników
konferencji, ale sytuacja całościowo wygląda inaczej dlatego warto o niej
wspomnieć.

Są hotele, które utrzymują się praktycznie tylko z organizowania konferencji, a
pobyty dla osób indywidualnych są bardzo małą częścią zysku hotelu. Przykładem
takiego hotelu jest hotel Gromada w Warszawie, który posiada największą bazę sal
konferencyjnych w Warszawie, a przynajmniej jedną z większych.

Jak już wspomniałem o salach konferencyjnych to jest to właśnie nieodzowny
element konferencji. Oprócz tego trzeba ulokować jeszcze uczestników
konferencji w pokojach. Zazwyczaj jest tak, że pokoje dla uczestników
konferencji są obok siebie. Nie jest to reguła.

W niektórych hotelach, na czas konferencji lub innego grupowego wydarzenia
uruchamiane jest specjalne stanowisko w recepcji, które obsługuje tylko
uczestników danej konferencji. Przykładem takiego hotelu może być hotel Marriott
w Warszawie. Wszystko zależy od wielkości grupy i szczegółów umowy jaka jest
podpisana między hotelem, a organizatorem konferencji. Nie mniej jednak
możliwość osobnego stanowiska istnieje. 

Przypomnę jeszcze, że każdy uczestnik konferencji ma własną rezerwację.

\subsection{Usługi}
Usługi, pod tym pojęciem kryje się wiele. Sekcję tą poświęcam na opisanie tych
najczęściej zamawianych oraz nakreślam jakie inne usługi mogą być oferowane
abstrahując od takich jak trufle na złotej tacy podane do łóżka.

\subsubsection{Definicja usługi}
Usługa to pewna korzyść, z której odpłatnie lub nie może skorzystać klient
podczas swojego pobytu. Pod to pojęcie można by podciągnąć bardzo wiele rzeczy.
Przykładowo korzystanie z basenu hotelowego, siłowni lub spa. Chciałbym się
jednak skupić nie na usłudze w sensie bogatej oferty hotelowej, a na tych
usługach, które wymagają specjalnego traktowania przez obsługę hotelu innego niż
dopisania tego do rachunku.
\subsubsection{Przykłady usług}
Przykładem takiej usługi, która może być przydatna dla zwykłego człowieka jest
budzenie. Zamawiamy w recepcji budzenie na godzinę taką i taką i mamy nadzieje,
że zostaniemy o tej godzinie obudzeni. Czy to nastąpi poprzez telefon czy
stukanie do drzwi to już zależy od wewnętrznej organizacji.

Innym przykładem usługi, może być dostarczenie klientowi np. świeżych elementów
wyposażenia łazienkowego takiego jak ręcznik albo kosmetyki. Takie usługi są
zazwyczaj dodatkowo płatne. Oczywiście na stanie dostępne są ręczniki i
podstawowe kosmetyki. Ta usługa natomiast dotyczy dodatkowych przyborów.

Kolejna usługa to np. dostarczenie do pokoju koja oraz kuwety dla naszego
pupila.

Zdarza się również, że za usługę dostępu do internetu trzeba zapłacić. Zatem
zależnie od hotelu można to traktować albo jako usługę dodatkową albo jako
ofertę hotelu wliczoną w cenę.

Nie sposób jest wymienić wszystkich możliwych usług, ponieważ wszystko zależy od
wyobraźni klienta i tego czy obsługa hotelu może i chce sprostać zadaniu. A jak
wiemy powinniśmy robić wszystko, aby zadowolenie klienta było jak największe.
Oczywiście wszystko ma swoją cenę, a cenę za niestandardowe rzeczy ponosi
klient.

\subsubsection{Klasyfikacja usług}
Można by zatem spróbować sklasyfikować usługi biorąc pod uwagę to czy system
informatyczny może pomóc w ich zrealizowaniu czy też nie. Czynnika ludzkiego
nie sposób jest wyeliminować w spełnianiu zachcianek klienta, tak więc tutaj
odgrywa on rolę szczególną.

System na pewno może pomóc w zapamiętaniu kto zamawiał budzenie i na kiedy i
odpowiednio przypomnieć o tym obsłudze hotelu lub samemu wykonać telefon jeśli
dysponujemy odpowiednią infrastrukturą. Zatem na pewno warto uwzględnić
podsystem notyfikacji. Natomiast w jaki sposób system może pomóc w przypadku gdy
klient zapyta o coś, czego nawet teraz w momencie pisania nie jestem w stanie
określić. Na pewno warto zapewnić możliwość dołączenia pewnych notatek do
profilu klienta np. informujących o tym, że klient posiada kota albo psa. Wtedy
następnym razem gdy klient nas odwiedzi będzie zaskoczony, że potrzebne
wyposażenie ma już na miejscu. Można sobie łatwo wyobrazić zadowolenie klienta z
faktu, że ktoś o nim pamiętał. 

\subsection{Cennik}
Ceny za pobyt i inne usługi biorą się z cennika. Cennik określany jest przez
właściciela hotelu lub osobę, na którą to zadanie zostało oddelegowane. Skupię
się jednak tutaj na cenach za pobyt i różnych jej wariantach.

\subsubsection{Najwyższa stawka}
W branży hotelarskiej występuje pojęcie \emph{Rack rate}, które określa pełną
cenę za pokój, którą by musiał zapłacić klient jeśli przyszedł by do hotelu i
chciał zamówić pokój.
Pojęcie to nie jest tłumaczone na język polski i w hotelach występuje po prostu
pod nazwą RACK. Można to nazwać ceną na wejściu. Jest to najwyższa stawka jaką
trzeba zapłacić i taką płaci klient indywidualny jeśli nie dokona rezerwacji. Na
pewno mniej zapłacimy jeśli dokonamy rezerwacji przez jakiegoś agenta z branży turystycznej.
Nawet jeśli dokonamy rezerwacji samemu to i tak możemy płacić wysoką cenę,
ponieważ nie ma żadnych przepisów, które stanowiły by inaczej.

\subsubsection{Kalendarz}
Na każdy dzień w roku określona jest cena podstawowa, która jest czynnikiem ceny
końcowej. Określone są także ceny za inne usługi.

\subsubsection{Czynniki ceny}
Jeśli dokonujemy rezerwacji to na cenę pobytu będzie składało się kilka
elementów. Potencjalne składowe:
\begin{itemize}
  \item czy weekend czy dzień roboczy,
  \item kategoria pokoju,
  \item dodatkowe usługi wymienione na rezerwacji,
  \item rabaty,
  \item cena podstawowa na dany dzień,
  \item rodzaj klienta.
\end{itemize}

\subsubsection{Cena wolna}
Cena jest wyliczana na podstawie składowych wymienionych powyżej, ale zawsze
istnieje możliwość określenia dowolnej ceny wpisanej ręcznie przez pracownika
do rezerwacji.
Taka cena określana jest mianem ceny wolnej.

\subsection{OpenTravel}
OpenTravel to organizacja non-profit, założona w \mbox{1999 r.} przez firmy
z branży turystycznej, której głównym zadaniem jest stworzenie struktur
wiadomości elektronicznych w celu ułatwienia komunikacji pomiędzy różnymi systemami z
branży turystycznej. OpenTravel jest złożone z firm reprezentujących linie
lotnicze, wypożyczalnie samochodów, hotele, linie żeglugowe, firmy techniczne i
dystrybutorów. Dziesiątki tysięcy struktur wiadomości jest używanych do
przesyłania dziesiątek milionów wiadomości pomiędzy firmami partnerskimi każdego
dnia.

Do członków OpenTravel należą sieci hotelarskie, a niektóre z nich to:
\begin{itemize}
  \item Hilton Hotels Corporation
  \item InterContinental Hotels Group
  \item Marriott International
\end{itemize}

Wymieniam je ponieważ mają one swoje hotele w Polsce. 

\subsubsection{Forma standardu}
Wiadomości z OpenTravel można traktować jako standard, do którego trzeba się
dostosować jeśli mamy konkurować na rynku światowym z innymi hotelami. Jest to
sposób na integracje całego spektrum biznesów branży turystycznej. Napisałem o
tej organizacji dlatego, żeby wskazać jej istnienie i to, że należało by je brać
pod uwagę jeśli chcielibyśmy napisać system dla ogólno światowej sieci hoteli
lub innego biznesu z zakresu turystycznego bądź transportowego.

Forma standardu to schematy XML'owe, które dokładnie określają każdy aspekt
przesyłanych wiadomości. Zainteresowanych czytelników odsyłam do źródła TODO
DODAĆ LINK DO ŹRÓDŁA / BIBLIOGRAFI
przydatny link: http://adriatic.pilotfish-net.com/ota-modelviewer/

\subsubsection{Dokumentacja standardu}
OpenTravel dla zarejestrowanych użytkowników z statusem członka udostępnia
przydatną dokumentację pokazującą jak wdrażać systemy oparte o ów wiadomości.
Jest to również przydatna lektura na temat WebServices jako takich. Osobiście
udało mi się otrzymać status członka i miałem okazję zapoznać się dostępną
dokumentacją.

Informacja dla zainteresowanych: o darmowy dostęp mogą ubiegać się pracownicy
naukowi oraz studenci.

\section{Literatura}

ustawia z dnia 29 sierpnia 1997r. o usługach turystycznych



%%% KONIEC ROZDZIAŁU PIERWSZEGO %%%%%

\appendix

% tutaj załączniki

%\chapter*{Bibliografia}
\nocite{*}
\bibliographystyle{plplain}
%\bibliographystylebk{plplain}
%\bibliographystylest{plplain}
%\bibliographystyledoc{plplain}
% \bibliographystyleweb{plplain}
%\bibliographybk{BIB/books}
%\bibliographyst{BIB/books}
%\bibliographydoc{BIB/books}
% \bibliographyweb{BIB/books}

% \bibliography{bib/verificard,bib/jml,bib/daikon}
\bibliography{../bib/daikon,../bib/statistics,../bib/other}

\end{document}

% ex: set tabstop=4 shiftwidth=4 softtabstop=4 noexpandtab fileformat=unix filetype=tex spelllang=pl,en spell:

